It is very difficult to imagine a world without electricity. For over one hundred years we are using it to accomplish a wide range of tasks. Yet still our civilization does not possess a reliable and cheap source of this energy for wide use. Nuclear fission has proved to be a good candidate. However, as a side effect it produces large amounts of nuclear waste. Nevertheless, scientist are working on a better alternative - Fusion. Unlike in fission, where heavy atoms are split into smaller atoms and thereby releasing energy, the process of fusion is quite the opposite. Light elements are forced to combine into heavier elements releasing vast amounts of energy. Scientist have discovered this process together with fission. Yet it proved very difficult to duplicate the process in an efficient and controlled manner. 

One of the key problem at reproducing fusion in an artificial environment are the extreme conditions required. The atoms naturally repel due to the coulomb force. Only after forcing the atoms close together does the strong nuclear force take over and fuse the atoms. In our sun, gravity is responsible for this. However on Earth, we are unable to create such enormous pressures. The only alternative are temperatures as high as millions of degrees where the kinetic energy of the atoms is high enough to bring the atoms close to one another. No element can withstand such temperatures. Therefore maintaining fusion over extensive periods of time requires finding a method of containing it safely in a reactor without touching the walls. 

Tokamak is a technology used to contain plasma in a vacuum ring using electromagnetic coils. Most fusion reactions take place in the plasma state of matter. This makes a Tokamak an ideal device for containing fusion. However to make sure the fusion is safely contained in the reactor one must first understand the process of plasma formation in the tokamak chamber.

The plasma formation, also known as the plasma breakdown or the avalanche effect, is one of the strangest parts of plasma physics. It very much depends on the initial conditions of the experiment and thus it is difficult to simulate. However it is possible to estimate, based on previous experiments, if the breakdown of the plasma is going to be successful given some initial conditions. This essay will focus at investigating methods for \textbf{ Predicting the breakdown probability for a plasma discharge in the GOLEM tokamak} 