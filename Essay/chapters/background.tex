\section{Rokamak background}

Tokamak is an abbreviation for 'toroidal chamber with magnetic coils'.
The history of the design reaches as far back as the 1950s when it was invented by the Soviet physicist Igor Tamm and Anrei Sakharov. 

%Add more info about the tokamak
The GOLEM tokamak situated at the CVUT faculty in Prag is


\section{Plasma creation process}
A tokamak consist of a toroidal chamber which is surrounded by coils. When current flows through the coils a toroidal magnetic field is generated inside of the chamber. This can be used to attract the ionized gas inside of the chamber towards the center and away from the walls.


The chamber of the GOLEM tokamak is surrounded by an iron core of a transformer. It is positioned in such a way as to induce electric current in the area of the chamber. If partially ionized gas is present in the chamber, its loose electrons are accelerated resulting in further ionization and rapid heating of the gas.

This plasma breakdown process is sometimes called the avalanche effect because the entire volume of the gas ionizes initiated by a small number of initial ions in the gas.

Whether the plasma breakdown is successful or not depends very much on the initial conditions. This includes the strength of the toroidal magnetic field, strength of the electric field, gas type and pressure, type and duration of the pre-ionization and many more.

This experiment will focus at estimating the probability of a successful breakdown given initial conditions. 

\section{Paschen's Law}
Paschen's Law applies to the breakdown voltage of a gas between two electrodes. It predicts the minimal voltage required to create an arc. According to Paschen's Law, the breakdown voltage of a gas is dependent mainly on two factors. The first being the pressure $p$ and the second the distance between the electrodes $d$. The breakdown voltage $V_{b}$ is given by the following formula:

$$ V_{b}={10}{x} $$